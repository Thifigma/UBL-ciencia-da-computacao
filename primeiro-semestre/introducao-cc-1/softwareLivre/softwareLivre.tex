\documentclass[12pt, onecolumn]{report}

\usepackage[utf8]{inputenc}
\usepackage[brazil]{babel}
\usepackage[top=2cm, bottom=2cm, right=2cm, left=2cm]{geometry}

\title{Software Livre}
\author{Thiago Figueiredo Marcos}
\date{\today}

\begin{document}

    \maketitle

    \tableofcontents

    \chapter{Introdução}
        \hspace{2cm} O texto que será aqui descrito é um breve relato sobre meu aprendizado na Universidade Brasileira Livre.
            Hoje o aprendizado é {\tt Software Livre.}

    \chapter{História e Opnião}
        \hspace{2cm} Seu aspecto histórico se inicia na decada de 80 e oficialmente em 1983 com o lançamento do projeto GNU.
            O principal intuito do movimento é disseminar o uso livre e ético de soluções há qualquer usuario. 
            e incentivo a abertura do código fonte. O movimento possui raizes filosóficas e impacto social, a própria UBL é um 
            exemplo concreto do movimento software livre, evidênciado pelo seu compartilhamento ilimitado do conhecimento a toda comunidade,
            além de aceitar contribuições, e sugestões de melhoria da própria comunidade. Observa-se que há nesta comunidade uma cultura ao compartilhamento.
            Devido a todas essas caracteristicas observa-se um nivel avançado de movimento, prezando sempre pelo melhoramento ético, técnico e público de técnologias.        
    
    \chapter{Definição}
        \begin{itemize}
            \item Liberdade para {\bf executar} o programa.
            \item Liberdade para {\bf estudar e modificar} o programa.
            \item Liberdade para {\bf distribuição} do programa.
            \item Liberdade para {\bf alteração e redistribuição} do programa.
        \end{itemize}

    \chapter{Exemplos de Softwares Livres}
        \begin{itemize}
            \item Python
            \item Java
            \item Ruby
            \item Android
            \item Firefox
            \item Kernel Linux
        \end{itemize}
        
    \chapter{Conclusão}
        \hspace{2cm} Concluimos que a vantagem em se adotar esse tipo de cultura é inegavel devido a diversidade de contribuições e qualidade do software.

\end{document} 
