\documentclass[12pt, onecolumn]{article}

\usepackage[utf8]{inputenc}
\usepackage[brazil]{babel}
\usepackage[top=2cm, bottom=2cm, right=2cm, left=2cm]{geometry}

\usepackage{amsmath}
\usepackage{graphicx}


\title{Software Livre}
\author{Thiago Figueiredo Marcos}
\date{\today}

\begin{document}

    \maketitle

    \tableofcontents

    \section*{Introdução}
        \hspace{2cm} O texto que será aqui descrito é um breve relato sobre meu aprendizado na Universidade Brasileira Livre.
            Hoje o aprendizado é {\tt Software Livre.}

    \section{História}
        \hspace{2cm}    Seu aspecto histórico se inicia na década de 80 e oficialmente em 1983 com o lançamento do projeto GNU.
            O principal intuito do movimento é disseminar o uso livre e ético de soluções 
            há qualquer usuario e incentivo a abertura do código fonte.\\ 

        \hspace{2cm}    Em 1989 foi fundada a Free Soft Foundation(FSF) por Richard Stalmann e foi criada a Licença Publica Geral ou (GLP),
            na sigla em inglês. A ideia geral de todo o projeto era criar um sistema operacional totalmente livre.\\

        \hspace{2cm}     Em 1991 Linus Torvalds lançou a primeira versão do Kernel do Linux, resultando em um sistema operacional livre
            foi construido utilizando as ferramentas do projeto GNU.

        \subsection{Opnião}
            \hspace{2cm}    O movimento possui raizes filosóficas e impacto social e econômico profundos. 
                A própria UBL é um exemplo concreto do movimento software livre, evidênciado pelo seu compartilhamento livre e ilimitado 
                do conhecimento a toda comunidade. \\ 
            \hspace{2cm} Observa-se que há nesta comunidade uma cultura forte ao compartilhamento do conhecimento e ajuda mutua.
                Devido a todas essas caracteristicas eu observo um nivel avançado de movimento, que  preza sempre pelo melhoramento 
                ético, técnico e público de técnologias além do compartilhamento do conhecimento.
        
        \subsection{Empreendedorismo}
            \hspace{2cm}    Ao empreendedor adotar práticas e uma cultura de software livre em sua empresa pode trazer diversos 
                beneficios, como redução de custos com manutenção de códigos, e alta resolutividade de problemas de forma criativa
                e com uma alta qualidade. Ao exibir o código fonte da solução o empreendedor fica vulneravel a criticas das mais diversas
                inclusive as construtivas, tornando sua solução única, mesmo que outro copie o código fonte, é absoleto para o mesmo problemas
                pois em tese, teria toda uma comunidade já envolvida na solução original. \\
            \hspace{2cm}    Eu observo por exemplo no Brasil, em como soluções de software livre poderiam impactar o Sistema Unico de Saúde
                principalmente na atenção primária, onde há mais necessidade, visto que o SUS possui semelhanças com o Software Livre.
                Obsreve que qualquer conjunto de poder, pode copiar o SUS para sua população.

    \section{Definição}
        \begin{itemize}
            \item Liberdade para {\bf executar} o programa.
            \item Liberdade para {\bf estudar e modificar} o programa.
            \item Liberdade para {\bf distribuição} do programa.
            \item Liberdade para {\bf alteração e redistribuição} do programa.
        \end{itemize}

    \section{Exemplos de Softwares Livres}
        \begin{itemize}
            \item Python
            \item Java
            \item Ruby
            \item Android
            \item Firefox
            \item Kernel Linux
        \end{itemize}
        
    \section{Conclusão}
        \hspace{2cm} Concluimos que a vantagem em se adotar esse tipo de cultura é inegavel devido a diversidade de contribuições 
        e qualidade do software.

\end{document} 
