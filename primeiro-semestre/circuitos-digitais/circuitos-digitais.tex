\documentclass[12pt, onecolumn]{article}

\usepackage[utf8]{inputenc}
\usepackage[brazil]{babel}
\usepackage[top=2cm, bottom=2cm, right=2cm, left=2cm]{geometry}


\usepackage{graphicx}
\usepackage{natbib}


	\title{Circuitos Digitais}
	\author{Thiago Figueiredo Marcos}
	\date{\today}

\begin{document}
	
	\maketitle

	\section{Introdução}
	Um computador digital pode ser descrito como aquilo que computa, ou aquilo 
	que processa informação digital. A informação que é processada por um circuito 
	digital é aquela que é \textbf{quantizada} ou \textbf{discretada} \citep{art1}.\\
	\\
	No mundo comum, as informações são analógicas, ou seja, a onda que representa 
	aquela informação possui uma gama de valores diferenciados. Já os sinais digitais
	podem ser representados apenas por dois valores, ou seja, uma lógica binária.\citep{bk1}\\ 
	\\
	Os circuitos digitais são construidos por componentes eletrônicos e tem como
	entradas e saidas sinais digitais. Para usarmos informações do mundo analógico
	é preciso discretar essas informações, afím de convertelas à binária e geralmente é
	usado uma medida em Volts para determinar se uma informação é ligada ou desligada. 
	Após convertida a informação e processada no circuito digital é preciso converter
	o sinal de saída do circuito que é digital, em analógico novamente.\citep{art2}

	\section{Conversão Analógico - Digital (Discretação)}
	Sinais discretos são frequências descontinuas no tempo, ou seja, definida apenas
	para determinados instantes. Representa aproximadamente o mundo real, entretanto,
	podem ser utilizadas várias técnicas para melhorar a representação, como as de 
	processamento de sinais digitais.\\
	\\
	Aqui também vale ressaltar que o processo de discretação de alguma informação
	está consequentemente ligada a perca de determinadas informações.
		
		\subsection{Principais propiedades da discretação}
		\begin{enumerate}
			\item\textbf{Amostragem}: Discretação do sinal analógico no tempo.
			\item\textbf{Quantização}: Discretação da amplitude do sinal amsotrado 
				em niveis.
			\item\textbf{Codificação}: Atribuição de códigos, onde geralmente 
				são binários às	amplitudes do sinal quantizado.
		\end{enumerate}

	\section{Conversão Digital - Analógico (Linearização)}
	Se refere a o processo que transforma um sinal modelado por eventos discretos
	em um sinal contínuo, ou seja, o processo de integração de vários sinais discretos
	para simular um evento contínuo.

	\section{Processamento}
	O processamento de informação se refere a diversas operaçõs realizadas por um circuito
	digital para transformar a entrada de dados em uma saida significativa de interesse. 
	Isso pode ser calculos, manipular dados como agregação, separação e classificação 
	ou ainda filragem, entre outros. Além disso, no circuito digital é simplificado o 
	armazenamento de informações bem como possui uma menor probabilidade de interferências.

		
	% Referências. 

	\bibliographystyle{apalike}
        \bibliography{ref.bib}

\end{document}
