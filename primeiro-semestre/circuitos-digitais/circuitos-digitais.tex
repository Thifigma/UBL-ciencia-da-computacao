\documentclass[12pt, onecolumn]{article}

\usepackage[utf8]{inputenc}
\usepackage[brazil]{babel}
\usepackage[top=2cm, bottom=2cm, right=2cm, left=2cm]{geometry}


\usepackage{graphicx}
\usepackage{natbib}


	\title{Circuitos Digitais}
	\author{Thiago Figueiredo Marcos}
	\date{\today}

\begin{document}
	
	\maketitle

	\section{Introdução}
	Um computador digital pode ser descrito como aquilo que computa, ou aquilo 
	que processa informação digital. A informação que é processada por um circuito 
	digital é aquela que é \textbf{quantizada} ou \textbf{discretada} \citep{art1}.\\
	\\
	No mundo comum, as informações são analógicas, ou seja, a onda que representa 
	aquela informação possui uma gama de valores diferenciados. Já os sinais digitais
	podem ser representados apenas por dois valores, ou seja, uma lógica binária.\citep{bk1}\\ 
	\\
	Os circuitos digitais são construidos por componentes eletrônicos e tem como
	entradas e saidas sinais digitais. Para usarmos informações do mundo analógico
	é preciso discretar essas informações, afím de convertelas à binária e geralmente é
	usado uma medida em Volts para determinar se uma informação é ligada ou desligada. 
	Após convertida a informação e processada no circuito digital é preciso converter
	o sinal de saída do circuito que é digital, em analógico novamente.\citep{art2}

	\section{Conversão Analógico - Digital (Discretação)}
	Sinais discretos são frequências descontinuas no tempo, ou seja, definida apenas
	para determinados instantes. Representa aproximadamente o mundo real, entretanto,
	podem ser utilizadas várias técnicas para melhorar a representação, como as de 
	processamento de sinais digitais.\\
	\\
	Aqui também vale ressaltar que o processo de discretação de alguma informação
	está consequentemente ligada a perca de determinadas informações.
		
		\subsection{Principais propiedades da discretação}
		\begin{enumerate}
			\item\textbf{Amostragem}: Discretação do sinal analógico no tempo.
			\item\textbf{Quantização}: Discretação da amplitude do sinal amsotrado 
				em niveis.
			\item\textbf{Codificação}: Atribuição de códigos, onde geralmente 
				são binários às	amplitudes do sinal quantizado.
		\end{enumerate}

	\section{Conversão Digital - Analógico (Linearização)}
	Se refere a o processo que transforma um sinal modelado por eventos discretos
	em um sinal contínuo, ou seja, o processo de integração de vários sinais discretos
	para simular um evento contínuo.

	\section{Processamento}
	O processamento de informação se refere a diversas operaçõs realizadas por um circuito
	digital para transformar a entrada de dados em uma saida significativa de interesse. 
	Isso pode ser calculos, manipular dados como agregação, separação e classificação 
	ou ainda filragem, entre outros. Além disso, no circuito digital é simplificado o 
	armazenamento de informações bem como possui uma menor probabilidade de interferências.
	
	\section{Sistemas de numeração}
	Número nos remete a ideia de quantidade, já o numeral é a representação desta 
	ideia, na prática, nos referimos a palavra número para qualquer tipo de
	representação numeral.\\
	\\
	\textbf{Exemplo}: A quantidade \textbf{Quarenta e dois} é representada pelo
	numeral 42.\\
	\\
	Sem o conhecimento da organização posicional dos números, como podemos
	representar todos eles? Poderiamos pensar em um simbolo para cada número,
	porém, existe uma infinidade de quantidades.\\
	\\
	Há cerca de 3.000 anos atrás os \textbf{Egípcios} desenvolveram um sistema
	de numeração, entre esse sistema esta a base 10, na qual utilizamos até hoje.
	Os números representados por hieróglifos eram mais usados em monumentos e 
	templos, pintados ou talhados em pedra. 
	Há sete símbolos, representando os números 1, 10, 100, 1000, 10 000, 
	100 000 e 1 000 000.\citep{art3}\\
	\\
	Algarismos é um conjunto finito de símbolos numéricos que usamos para
	representar quantidades reais. Todo e qualquer número pode ser representado
	por uma combinação de algarismos, os mais conhecidos são os indo-arábicos:
	0, 1, 2, 3, 4, 5, 6, 7, 8, 9.\\
	\\
	Sistema de numeração é a forma de atribuir uma representação única para cada
	número. O sistema de numeração posicional atribui valor ao algarismo conforme
	a sua posição, mais a esquerda ou mais a direita.
	\\
	No sistema decimal de numeração posicional possuimos 10 algarismos, 0 .. 9,
	um cada um deles representa seu valor absoluto, ou seja, o valor 0 representa
	o nada, o valor 1 representa uma única unidade e assim por diante. Dependendo
	da posição que o valor estiver, seu valor absoluto pode variar, por exemplo:
	Imagine um número com 4 casas decimais, - - - -, o número que estiver na 
	"casa" mais a esquerda não representara seu valor absoluto e sim o seu valor
	absoluto multiplicado por um milhão.\\
	\\
	\textbf{Exemplo}: 4237 = 4*1000 + 2*100 + 3*10 + 7*1, observe o seguinte:\\
	\\
	4*1000 	= 4.000\\
	2*100 	= 200\\
	3*10	= 30\\
	7*1	= 7\\
	\\
	A soma de todos os valores resulta no valor original 4237.\\
	\\
	Agora podemos sistematizar isso matematicamente, sabemos que um número
	inteiro A no sistema decimal é presentado por N digitos assim:\\
	\begin{center}
		$A_{n-1} A_{n-2} ... A_{n2} A_{n1} A_{n0}$
	\end{center}
	Cada $a_{i}$ é um algarismo decimal.

	\begin{center}
		$a_{n-1}*10^{n-1} + a_{n-2}*10^{n-2}+ ... + 
		a_{2}*100 + a_{1} * 10 + a_{0} * 1$
	\end{center}

	ou seja,

	\begin{center}
		$\sum_{i = 0}^{n - 1} a_{i} * 10^{i}$
	\end{center}

	Usando a mesma lógica, podemos representar os números racionais no sistema
	decimal
	
	\begin{center}
		$\sum_{i = 0}^{n - 1} a_{i} * 10^{i} + 
		\sum_{i = 1}^{\infty} a_{-i} * 10^{-i}$
	\end{center}

	\section{Truncamento}
	Vemos que à medida que caminhamos mais a direita depois da virgula, o
	valor relativo a cada algarismo se torna cada vez menor, podemos
	fazer uma representação aproximada do número gerado, limitando 
	o número de algarismos após a vírgula por uma constante \texttt{M}.\\
	\\
	Essa aproximação chama-se truncamento. Com o truncamento há também
	um erro de aproximaçaõ que pode ser obtido com a diferença do número original
	com o número truncado.\\
	\\
	Ao truncarmos um número com uma constante \texttt{M} para qualquer 
	número real com n algarismos à esquerda da virgula, 
	e \texttt{M} algarismos à direita, assim: 

	\begin{center}
		$a_{n-1} a_{n-2} ... a_{1} a_{0}, a_{-1} a_{-2} ... a_{M}$
	\end{center}

	então temos que o erro de aproximação de qualquer número N será:

	\begin{center}
		$\texttt{err} < 10^{-M}$
	\end{center}
	
	ou seja, aumentar \texttt{M} implica em diminuir o erro.


	% Referências. 

	\bibliographystyle{apalike}
        \bibliography{ref.bib}

\end{document}
