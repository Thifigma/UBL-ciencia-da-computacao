\documentclass[12pt, onecolumn]{article}

\usepackage[utf8]{inputenc}
\usepackage[brazil]{babel}
\usepackage[top=2cm, bottom=2cm, right=2cm, left=2cm]{geometry}


\title{Linguagens de Programação}
\author{Thiago Figueiredo Marcos}
\date{\today}


\begin{document}

	\maketitle
	
	\begin{abstract}
		Essa disciplina será baseada no livro: \texttt{Conceito de 
		Linguagens de Programação} de \textbf{Robert Sebesta}. Utilizei
		a bibliteca digital da UFPR para acessar o exemplar, além de vídeos
		aulas do Prof. Dr. Filipe Braida no youtube.
	\end{abstract}


	\section{Capitulo 1 - Preliminares}
	O autor relata diversos motivos pelos quais um estudante de ciência da
	computação deveria se debruçar sobre conceitos gerais de linguagens de
	programação, entre eles, a capacidade de expressar a complexidade dos 
	pensamentos e também o maior acesso a uma diferenciada gama de ferramentas
	para resolver diversos tipos de problemas diferentes.
		
		\subsection{Domínios de programação}
		Aqui o autor relata as diversidades das aplicações dos 
		computadores modernos, e como as linguagens de programação
		podem ser construidas por motivos e aplicabilidades diferentes.
		Deixo abaixo alguns destaques: 

			\begin{enumerate}
				\item {Aplicações Científicas}
				\item {Inteligência Artificial}
				\item {Softwares para a WEB}
			\end{enumerate}

		\subsection{Critérios de avaliação de LPs}
		Nesta seção é levantado alguns critérios genéricos para 
		avaliar uma LP, essas características serão análisadas com
		maior profundidade mais para frente, de momento, será passado
		um entendimento geral para caracterizar uma determinada linguagem.\\
		\\
		A facilidade com o qual é feita a manutenção em um software é 
		determinada pela legibilidade do código, e esta, se torna uma medida 
		importante para determinar a qualidade do programa e da linguagem.\\
		\\
		Os critérios que são levados em conta ao caracterizar uma linguagem 
		são os seguintes: 

		\begin{enumerate}
			\item {Legibilidade}
			\item {Facilidade de escrita}
			\item {Confiabilidade}
			\item {Custo}\\
		\end{enumerate}
		Esses valores não necessáriamente é posto em grau de importância, mas 
		sim de consenso para avaliação.\\
		\\
		A simplicidade geral da linguagem também afeta sua legibilidade, uma
		linguagem com muitos operadores ou ainda, com diversas maneiras de
		fazer uma operação, pode ternala uma linguagem de pouca legibilidade.
		A sobrecarga de operadores, ou seja, multiplos significados em um único
		operador reduz a legibilidade, porém nem sempre a simplicidade pode
		contribuir com a legibilidade dos programas, um exemplo é a linguagem
		assembly.\\
		\\
		A ortogonalidade também está relacionada à simplicidade, quanto mais
		ortogonal for uma linguagem maior as combinações e manipulações dos
		tipos primitivos e menor o numero de exceções ás regras da linguagem.\\
		\\
		Por fim, o estudo de linguagens de programação aumenta nossa capacidade
		de usarmos diferentes construções ao escrevermos programas, nos 
		permite escolher uma linguagem com menores chances de maiores custos,
		além de facilitar o aprendizaddo de novas linguagens.


	\section{Nomes, vinculações e escopos}
	Aqui discutiremos as adversidades das variaveis em seus diferentes escopos, 
	onde e como essas variáveis são alocadas, além do tempo de vida.\\
	\\
	As linguagens imperativas, ou seja, aquelas que são executadas linha a linha,
	carregam em sí a arquitetura de \texttt{Von Neumann}, que consiste em 
	armazenar além dos dados as instruções do programa para posteriormente 
	serem fornecidas ao processador.\\
	\\
	Uma célula de memoria é um bloco fundamental da memoria do computador, ou seja,
	um circuito eletrônico que tem a capacidade de armazenar bits, 
	\texttt{isso é visto com mais detalhes na disciplina de circuitos digitais} em
	um programa, nós ciêntistas da computação, abstraimos esse circuito e a 
	chamamos de váriavel. Em uma arquitetura convencional de 64 bits um 
	\texttt{inteiro}, tem, geralmente, 4 bits. Em um alto nivel de abstração das
	células de memória um tipo de dado \texttt{inteiro} poderia armazenar 4 
	\texttt{char}, por isso uma variável possui propiedades e atributos para
	serem caracterizadas e distinguidas, ainda assim, em baixo nível, essas 
	atribuições podem ocorrrer, causando falhas no programa, sem que o compilador 
	reconheça.\\
	\\

	\subsection{Nomes}
	Aqui vou pular a parte de nomeação de variáveis, pois foi exaustivamente
	tratada na disciplina de introdução a ciência da computação. Nos restringiremos
	a tratar das palavras reservadas e o significado delas.\\
	\\
	Uma palavra reservada em uma linguagem de programação é uma palavra 
	\texttt{especial}, que não pode ser usada como nomes, logo, se uma
	linguagem possui um número alto de palavras reservadas, o programador
	terá que usar da criatividade para atribuir nomes.\\
	\\

	





\end{document}
