\documentclass[12pt, onecolumn]{article}

\usepackage[utf8]{inputenc}
\usepackage[brazil]{babel}
\usepackage[top=2cm, bottom=2cm, right=2cm, left=2cm]{geometry}


\title{Linguagens de Programação}
\author{Thiago Figueiredo Marcos}
\date{\today}


\begin{document}

	\maketitle
	
	\begin{abstract}
		Essa disciplina será baseada no livro: \texttt{Conceito de 
		Linguagens de Programação} de \textbf{Robert Sebesta}. Utilizei
		a bibliteca digital da UFPR para acessar o exemplar, além de vídeos
		aulas do Prof. Dr. Filipe Braida no youtube.
	\end{abstract}


	\section{Capitulo 1 - Preliminares}
	O autor relata diversos motivos pelos quais um estudante de ciência da
	computação deveria se debruçar sobre conceitos gerais de linguagens de
	programação, entre eles, a capacidade de expressar a complexidade dos 
	pensamentos e também o maior acesso a uma diferenciada gama de ferramentas
	para resolver diversos tipos de problemas diferentes.
		
		\subsection{Domínios de programação}
		Aqui o autor relata as diversidades das aplicações dos 
		computadores modernos, e como as linguagens de programação
		podem ser construidas por motivos e aplicabilidades diferentes.
		Deixo abaixo alguns destaques: 

			\begin{enumerate}
				\item {Aplicações Científicas}
				\item {Inteligência Artificial}
				\item {Softwares para a WEB}
			\end{enumerate}

		\subsection{Critérios de avaliação de LPs}
		Nesta seção é levantado alguns critérios genéricos para 
		avaliar uma LP, essas características serão análisadas com
		maior profundidade mais para frente, de momento, será passado
		um entendimento geral para caracterizar uma determinada linguagem.\\
		\\
		A facilidade com o qual é feita a manutenção em um software é 
		determinada pela legibilidade do código, e esta, se torna uma medida 
		importante para determinar a qualidade do programa e da linguagem.\\
		\\
		Os critérios que são levados em conta ao caracterizar uma linguagem 
		são os seguintes: 

		\begin{enumerate}
			\item {Legibilidade}
			\item {Facilidade de escrita}
			\item {Confiabilidade}
			\item {Custo}\\
		\end{enumerate}
		Esses valores não necessáriamente é posto em grau de importância, mas 
		sim de consenso para avaliação.\\
		\\
		A simplicidade geral da linguagem também afeta sua legibilidade, uma
		linguagem com muitos operadores ou ainda, com diversas maneiras de
		fazer uma operação, pode ternala uma linguagem de pouca legibilidade.
		A sobrecarga de operadores, ou seja, multiplos significados em um único
		operador reduz a legibilidade, porém nem sempre a simplicidade pode
		contribuir com a legibilidade dos programas, um exemplo é a linguagem
		assembly.\\
		\\
		A ortogonalidade também está relacionada à simplicidade, quanto mais
		ortogonal for uma linguagem maior as combinações e manipulações dos
		tipos primitivos e menor o numero de exceções ás regras da linguagem.\\
		\\
		Por fim, o estudo de linguagens de programação aumenta nossa capacidade
		de usarmos diferentes construções ao escrevermos programas, nos 
		permite escolher uma linguagem com menores chances de maiores custos,
		além de facilitar o aprendizaddo de novas linguagens.


	\section{Nomes, vinculações e escopos}
	Aqui discutiremos as adversidades das variaveis em seus diferentes escopos, 
	onde e como essas variáveis são alocadas, além do tempo de vida.\\
	\\
	As linguagens imperativas, ou seja, aquelas que são executadas linha a linha,
	carregam em sí a arquitetura de \texttt{Von Neumann}, que consiste em 
	armazenar além dos dados as instruções do programa para posteriormente 
	serem fornecidas ao processador.\\
	\\
	Uma célula de memoria é um bloco fundamental da memoria do computador, ou seja,
	um circuito eletrônico que tem a capacidade de armazenar bits, 
	\texttt{isso é visto com mais detalhes na disciplina de circuitos digitais} em
	um programa, nós ciêntistas da computação, abstraimos esse circuito e a 
	chamamos de váriavel. Em uma arquitetura convencional de 64 bits um 
	\texttt{inteiro}, tem, geralmente, 4 bits. Em um alto nivel de abstração das
	células de memória um tipo de dado \texttt{inteiro} poderia armazenar 4 
	\texttt{char}, por isso uma variável possui propiedades e atributos para
	serem caracterizadas e distinguidas, ainda assim, em baixo nível, essas 
	atribuições podem ocorrrer, causando falhas no programa, sem que o compilador 
	reconheça.\\
	\\

	\subsection{Nomes}
	Aqui vou pular a parte de nomeação de variáveis, pois foi exaustivamente
	tratada na disciplina de introdução a ciência da computação. Nos restringiremos
	a tratar das palavras reservadas e o significado delas.\\
	\\
	Uma palavra reservada em uma linguagem de programação é uma palavra 
	\texttt{especial}, que não pode ser usada como nomes, logo, se uma
	linguagem possui um número alto de palavras reservadas, o programador
	terá que usar da criatividade para atribuir nomes.\\
	\\
	\subsection{Endereços}
	O endereço de uma variável, é o endereço de cécula de memória na qual
	ela está associada. Em linguagens de programação o endereço da variável 
	é geralmente associado ao valor esquerdo de uma sentença, lembrando que, 
	é possivel que tenha tipos de variáveis que podem ser associadas a um mesmo 
	endereço de memória. Quando mais de uma variável é usada para acessar um 
	único endereço,	é necessário muito cuidado, pois, os valores podem ser 
	alterados por uma atribuição, o que também pode prejudicar a legibilidade 
	e manutençaõ do código. O momento no qual uma variável se associa a um
	endereço é muito importante e diz muito sobre a linguagem usada.
	
	\subsection{Tipos e Valores}
	O tipo de valor de uma variável, determina a faixa de valores que ela pode
	armazenar e o conjunto de operações que podem ser realizadas para esse tipo.
	Já o valor de uma variável se refere a o conteúdo da célula de memória. 
	Uma célula de memória, pode ter um espaço pré-definido, como por exemplo 
	1 byte, porém, são valores que podem não caber determinado dado, que pode 
	ter mais que 1 byte, logo, é conveniente, que pensemos em células de memória, 
	como espaços abstratos, que por sua vez, se adapta ao tamanho do dado.\\
	\\
	O valor da variável em muitos casos é chamado de lado direito.
	

	\subsection{Vinculações}
	Vinculação é uma associação entre um atributo e uma entiedade, para exemplificar
	pense no simbolo da operação de soma, \texttt{'automaticamente'}, sabemos que 
	é o simbolo de \texttt{( + )}, esse é um tipo de vinculação, outro exemplo seria
	uma declaração de tipo em C, como por exemplo \textbf{int} a, sabemos que a 
	variável a está associada a o tipo de dado inteiro, que por sua vez, está
	vinculado a uma faixa de valores.\\
	\\
	A grande questão é quando essas vinculações ocorrem, na sentença: \textbf{int} a
	o valor \textbf{int} será associado a uma faixa de valores em tempo de 
	projeto da linguagem, ou seja quando estivermos projetando a linguagem, 
	reservaremos a palavra \textbf{int} e a associaremos uma faixa de valor. 
	Já o valor \textbf{a}, é associado a o valor inteiro em tempo de compilação, 
	no caso da linguagem C.\\
	\\
	O entendimento completo de tempos de vinculação é fundamental para um real
	entendimento do funcionamento semântico de uma linguagem de programação.
	
		\subsubsection{\hspace{1cm}Vinculação Dinâmica}
	Este tipo de vinculação é feito quando uma célula de memória, recebe um dado,
	o tipo dado determinará o tipo da váriavel, ou seja, o tipo variável é 
	determinada pelo seu valor.

		\subsubsection{\hspace{1cm}Vinculação Estática}
	Este tipo de vinculação é feito quando determinamos o tipo de dado que 
	uma célula de memória vai receber, ou seja, o tipo da variável é determinado
	antes de receber o dado.\\
	\\
	A principal diferença entre elas é que a dinâmica da mais flexibilidade 
	para o programador, enquanto a estática, define a variável aquele tipo e 
	não pode mais ser alterada em tempo de execução, o que pode, evitar falhas
	no programa.
	
	\section{Tempo de vida de variáveis}
	Uma variável deve ser associada a uma céula de memória, essa célula deve
	ser obtida de um conjunto de células disponíveis, esse processo é chamado
	de \texttt{alocação}, quando disponibilizamos novamente essa célula
	para o conjunto de células disponiveis, esse processo chamamos de
	\texttt{liberação}. Dizemos que o tempo de vida de uma variável é 
	a diferença entre o tempo que ela foi liberada e alocada, ou seja, 
	enquanto ela está vinculada a uma célula de memória.\\
	\\
	Existem diversos tipos de vinculações e elas podem ocorrer em diferentes
	tempos, como por exemplo as variáveis \texttt{estáticas}, que são 
	vinculadas a uma célula de memória antes do inicio da execução do programa,
	ou seja, em tempo de compilação, isso pode gerar vantagens e desvantagens.
	Uma vantagem seria uma menor sobrecarga de tempo para alocações de memória
	em tempo de execução, uma desvantagem muito grande é a impossibilidade
	de criar funções recursivas.\\
	\\
	Existem outras variáveis que chamamos de dinâmicas, pois são vinculadas
	em tempo de execução e só podem ser referênciadas por ponteiros. 
	Esses tipos de variáveis devem ser alocadas e liberadas por instruções
	explicitas e em tempo de execução, em C por exemplo, temos a função 
	\texttt{malloc}	que faz esse tipo de alocação. \texttt{Reforçamos que é dever
	do programador liberar a memória alocada depois de alocada.}\\
	\\
	O principal problema em usar esse tipo de alocação é o cuidado com o 
	gerenciamento de memória da \texttt{heap} que se não for administrado
	da forma correta, podem ocasionar \texttt{overflow} entre outros erros.

	\section{Escopo}
	Um dos fatores para o entendimento das variáveis é o escopo. Uma variável
	é visível se ela pode ser referenciada ou atribuida em sentenças.
	Muitas linguagens permitem que sejam criados novos escopos estáticos no
	meio do código executável, esse escopo permite que tenha suas próprias
	variáveis locais, cujo escopo é restrito no bloco. Uma variável é local
	quando ela é alocada em um escopo.\\
	\\

\end{document}
