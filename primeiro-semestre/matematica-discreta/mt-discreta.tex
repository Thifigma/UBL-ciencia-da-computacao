\documentclass[12pt, onecolumn]{article}

\usepackage[utf8]{inputenc}
\usepackage[brazil]{babel}
\usepackage[top=2cm, bottom=2cm, right=2cm, left=2cm]{geometry}

\title{Matemática Discreta para Computação}
\author{Thiago Figueiredo Marcos}
\date{\today}

\begin{document}
	\maketitle
	
	\begin{abstract}
	Essa disciplina será baseada no livro: \texttt{Elementos da Matemática Discreta
		para computação} do Prof. Dr. \textbf{Jorge Stolfi},
		além das orientações em vídeo aula do no youtube do Prof. Dr. 
		\textbf{Rudini Menezes Sampaio}
	\end{abstract}

	\section{Lógica Proposicional}
	Uma proposição é uma sentença que pode assumir valores \texttt{Verdadeiro} 
	ou \texttt{Falso}, não é necessário que se saiba o valor da sentença, apenas
	que seja possivel atribuir algum desses dois valores.\\
	\\
	Sentenças que não são proposições, logicamente, não podem receber valores
	\texttt{Verdadeiros} ou \texttt{Falsos}, porém, observa-se que sentenças
	interrogativas, imperativas em geral não são proposições. Uma sentença 
	declarativa que tenha dependencia de variáveis pode ser considerada
	proposição, dês de que os valores das variáveis sejam definidos.
	
		\subsection{Conectivos lógicos e proposições compostas}
		Conectivos lógicos podem ser entendidos como: \texttt{e}, 
		\texttt{ou}, \texttt{não}, \texttt{se ... então}. Esses 
		conectivos permitem formar proposições compostas.\\
		\\
		Uma proposição composta, possui na sua estrutura, composições
		simples ou \texttt{atômica}.

		\subsection{Notação para cálculo proposicional}
		A lógica proposicional é um formalismo que nos permite
		determinar o valor lógico das proposições. As letras minúsculas
		será a representação das proposições. Abaixo descreveremos
		os sinais dos conectivos lógicos \texttt{(operadores)}.\\
		\\
		\begin{description}
			\item[Conjunção]: $p \land q$
			\item[Disjunção]: $p \lor q$
			\item[Negação]: $\lnot p$ ou ainda $\bar{q}$
			\item[Implicação]: $p \longrightarrow q$
			\item[Equivalência]: $p \Longleftrightarrow q$
			\item[Disjunção Exclusiva]: $p \oplus q$
		\end{description}

		A implicação é um dos mais importantes conectivos da lógica 
		matemática. Descreve-se da seguinte forma: 

		\begin{center}
			\texttt{Hipotese, premissa ou antecedente} 
			\textbf{Verdadeira} 
			$\longrightarrow$
			\texttt{Tese, conclusão ou consequência} 
			\textbf{verdadeira}
		\end{center}

		\subsection{Procedência dos operadores lógicos}	
		
		
\end{document}
