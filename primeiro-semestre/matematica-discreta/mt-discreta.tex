\documentclass[12pt, onecolumn]{article}

\usepackage[utf8]{inputenc}
\usepackage[brazil]{babel}
\usepackage[top=2cm, bottom=2cm, right=2cm, left=2cm]{geometry}

\title{Matemática Discreta para Computação}
\author{Thiago Figueiredo Marcos}
\date{\today}

\begin{document}
	\maketitle
	
	\begin{abstract}
	Essa disciplina será baseada no livro: \texttt{Elementos da Matemática Discreta
		para computação} do Prof. Dr. \textbf{Jorge Stolfi},
		além das orientações em vídeo aula do no youtube do Prof. Dr. 
		\textbf{Rudini Menezes Sampaio}
	\end{abstract}

	\section{Lógica Proposicional}

	Uma proposição é uma sentença que pode assumir valores \texttt{Verdadeiro} 
	ou \texttt{Falso}, não é necessário que se saiba o valor da sentença, apenas
	que seja possivel atribuir algum desses dois valores.\\
	\\
	Sentenças que não são proposições, logicamente, não podem receber valores
	\texttt{Verdadeiros} ou \texttt{Falsos}, porém, observa-se que sentenças
	interrogativas, imperativas em geral não são proposições. Uma sentença 
	declarativa que tenha dependencia de variáveis pode ser considerada
	proposição, dês de que os valores das variáveis sejam definidos.
	
		\subsection{Conectivos lógicos e proposições compostas}

	Conectivos lógicos podem ser entendidos como: \texttt{e}, \texttt{ou}, 
	\texttt{não}, \texttt{se ... então}. Esses conectivos permitem formar 
	proposições compostas.\\
	\\
	Uma proposição composta, possui na sua estrutura, composições simples ou 
	\texttt{atômica}.

		\subsection{Notação para cálculo proposicional}

	A lógica proposicional é um formalismo que nos permite determinar o valor 
	lógico das proposições. As letras minúsculas será a representação das 
	proposições. Abaixo descreveremos os sinais dos conectivos lógicos 
	\texttt{(operadores)}.\\
	\\
		\begin{description}
			\item[Conjunção]: $p \land q$
			\item[Disjunção]: $p \lor q$
			\item[Negação]: $\lnot p$ ou ainda $\bar{q}$
			\item[Implicação]: $p \longrightarrow q$
			\item[Equivalência]: $p \Longleftrightarrow q$
			\item[Disjunção Exclusiva]: $p \oplus q$
		\end{description}

	A implicação é um dos mais importantes conectivos da lógica matemática. 
	Descreve-se da seguinte forma: 

		\begin{center}
			\texttt{Hipotese, premissa ou antecedente} 
			\textbf{Verdadeira} 
			$\longrightarrow$
			\texttt{Tese, conclusão ou consequência} 
			\textbf{verdadeira}
		\end{center}

		\subsection{Procedência dos operadores lógicos}

	Em uma proposição que usa dois ou mais operadores lógicos a ordem em que são
	aplicados é importante. Podemos aplicar parenteses nas proposições para indicar
	a maior precedência. Também há regras para indicar a maior precedência entre
	os operadores: \\
	\\
		\begin{table}[h]
		\centering
			\begin{tabular}{|c|c|}
				\hline
				Operador & Precedência\\ \hline

				$\lnot$ & 1 \\
				$\land$ & 2 \\
				$\lor, \oplus$ & 3 \\
				$\longrightarrow, \Longleftrightarrow$ & 4 \\
				\hline
			\end{tabular}
		\end{table}
	
		\subsection{Tautologia e Contradições}
	Tautologia é uma proposição que é sempre verdadeira, para qualquer valor
	atômico que a componha.\\
	\\
	Pense na seguinte sentença: $P \lor \lnot p$, neste caso, \texttt{p}
	pode assumir qualquer valor que sua resposta será sempre verdadeira e
	isso é uma tautologia. Veremos como isso é aplicado diretamente na
	computação na disciplina de circuitos digitais em algebra boolena.\\
	\\
	Já a contradição é uma proposição composta que é sempre falsa, para
	qualquer valor atômico que a componha.\\
	\\
	Análogo a sentença da tautologia, porém com outro operador podemos
	exemplificar a contradição, observe: $p \land \lnot p$, ou seja, 
	p pode assumir qualquer valor, que sua proposição será sempre falsa.
	
		\subsection{Equivalência Lógica}
	Duas proposições são ditas equivalentes se possuirem valores lógicos iguais.
	Por exemplo: $p \Longleftrightarrow \lnot(\lnot p)$ ou seja, operações
	com valores tautológicos, chegam a equivalências.\\
		
		\subsubsection{Leis de equivalência}
	
	\texttt{Leis do elemento identidade}:\\
	$p \land V \rightarrow p$\\
	$p \lor F \rightarrow p$\\
	$p \leftrightarrow V \rightarrow p$\\
	$p \oplus F \rightarrow p$\\	
	\\
	\texttt{Leis da idempotência}:\\
	$p \land p \rightarrow p$\\
	$p \lor p \rightarrow p$\\
	\\
	\texttt{Leis da dominação}:\\
	$p \land V \rightarrow V$\\
	$p \lor F \rightarrow F$\\
	\\
	\texttt{Leis da comutatividade}:\\
	$p \land q \rightarrow q \land p$\\
	$p \lor q \rightarrow q \lor p$\\
	$p \oplus q \rightarrow q \oplus p$\\
	$p \leftrightarrow q \rightarrow q \leftrightarrow p$\\
	\\
	\texttt{Lei da redução ao absurdo}:\\
	$ -p \longrightarrow q \Rightarrow (p \land \lnot q) \longrightarrow F)$\\
	\\
	
	Existem outras leis como a de \texttt{De Morgan} que vai ser vista com
	profundidade na disciplina de circuitos digitais e não será comentada aqui.
	Outras como associatividade e distributivas também não será vista, pois, 
	segue a mesma lógica que operações aritméticas.

		\subsection{Síntese de proposições}






\end{document}
